\documentclass[14pt, a4paper]{extarticle}
\usepackage[left=2.5cm, right=1.5cm, top=2.5cm, bottom=2.5cm]{geometry}
\usepackage[utf8x]{inputenc}
\usepackage[T2A]{fontenc}
\usepackage[english,russian]{babel}

\usepackage{amsmath}
\usepackage{amssymb}
\usepackage[usenames]{color}

\title{Лабораторная работа 2 по дисциплине Системы аналитических вычислений, задание 3}
\author{Выполнил: Бирюков В.В., М8О-207Б-19}

\usepackage{hyperref}
\usepackage{sagetex}

\setlength{\sagetexindent}{10ex}

\renewcommand{\thesection}{\number\numexpr\value{section}-1\relax}
\renewcommand{\thesubsection}{\thesection.\number\numexpr\value{subsection}-1\relax}
\renewcommand{\thesubsubsection}{\thesubsection.\number\numexpr\value{subsubsection}-1\relax}

\setcounter{secnumdepth}{1}
\setcounter{section}{1}


\begin{document}

\section{Приведение уравнения поверхности второго рода к каноническому виду}

\begin{sagesilent}
    u(x, y, z) = 7*x^2 + 8*x*y + 3*y^2 + 8*x*z + 6*y*z + 3*z^2 + 6*x + y + 7
\end{sagesilent}

Уравнение поверхности второго порядка:
$$\sage{u(x, y, z) == 0}$$

\begin{sagesilent}
    A = matrix([
        [7, 4, 4],
        [4, 3, 3],
        [4, 3, 3]   
    ])
    B = vector([3, 0.5, 0])
    a0 = 7
\end{sagesilent}

Матрица квадратичной формы:

$A = \sage{A}$

Вектор линейной формы:

$B = \sage{B.n(digits=5)}$

Свободный член: $a_0 = \sage{a0}$

\begin{sagesilent}
    # Характеристический многочлен
    var('l', latex_name=r"\lambda")
    char_poly = (A - matrix.identity(3) * l).det().simplify_full()
\end{sagesilent}

Составим характеристический многочлен.

$$|A - \lambda * E| = \sage{char_poly}$$

\begin{sagesilent}
    # Собственные значения
    eigenvalues = [i.rhs() for i in solve(char_poly == 0, l)]
    eigenvalues.sort()
\end{sagesilent}

Найдем собственные значения -- корни характеристического многочлена.

$\lambda_1 = \sage{eigenvalues[0].n(digits=5)}, \lambda_2 = \sage{eigenvalues[1].n(digits=5)}, \lambda_3 = \sage{eigenvalues[2].n(digits=5)}$

\begin{sagesilent}
    eigenvectors = list()
    for i, ev in enumerate(eigenvalues, 1):
        evect = vector(RR, *(A - matrix.identity(3) * ev).right_kernel().basis())
        eigenvectors.append(evect)
\end{sagesilent}

Найдем собственные векторы.

$s_1 = \sage{eigenvectors[0].n(digits=5)}$

$s_2 = \sage{eigenvectors[1].n(digits=5)}$

$s_3 = \sage{eigenvectors[2].n(digits=5)}$

\begin{sagesilent}
    S = list()
    for ev in eigenvectors:
        S.append(ev / sqrt(ev * ev))
    S = matrix(S).T
\end{sagesilent}

Пронормируем собственные векторы и составим из них матрицу перехода.

$S = \sage{S.n(digits=5)}$

\begin{sagesilent}
    # Диагональная матрица
    A1 = S.T * A * S
\end{sagesilent}

Приведем матрицу квадратичной формы к диагональному виду.
$$\Lambda = S^T * A * S = \sage{A1.n(digits=5)}$$

\begin{sagesilent}
    # Преобразование коэффициентов линейной формы
    B1 = S.T * B
\end{sagesilent}

Преобразуем вектор линейной формы.

$B' = S^T * B = \sage{B1.n(digits=5)}$

\begin{sagesilent}
    # Почти приведенное уравнение
    v = a0
    variables = [x, y, z]
    for i in range(len(variables)):
        v += A1[i][i] * variables[i] ^ 2 + 2 * B1[i] * variables[i]
    show(v == 0)
\end{sagesilent}

Получим почти приведённое уравнение: $X \Lambda X^T + 2 B' X + a_0 = 0$

$\sage{v == 0}$

\begin{sagesilent}
    # Приведенное уравнение
    v = 0
    a1 = a0

    # Убираем линейные члены, приводя к полному квадрату, где это возможно
    for i in range(len(variables)):
        if A1[i][i] != 0:
            v += A1[i][i] * (variables[i] + B1[i] / A1[i][i]) ^ 2
            a1 -= B1[i] ^ 2 / A1[i][i]
        else:
            v += 2 * B1[i] * variables[i]

    v = (v + a1) / a1
\end{sagesilent}

Сделаем замену переменных, получим каноническое уравнение.

$\sage{v == 0}$

\pagebreak

График исходного уравнения

$\sage{u(x,y,z) == 0}$

\begin{center}
    \sageplot[trim=100 100 100 100, clip, width=10cm][png]{implicit_plot3d(u, (x, -10, 10), (y, -20, 20), (z, -20, 20), plot_points=100).rotateZ(2.5*pi/6)}
\end{center}

График канонического уравнения

$\sage{v == 0}$

\begin{center}
    \sageplot[trim=100 100 100 100, clip, width=10cm][png]{implicit_plot3d(v, (x, -20, 0), (y, -10, 10), (z, -10, 10), plot_points=100)}
\end{center}

\end{document}
